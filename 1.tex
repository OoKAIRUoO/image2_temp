\section{SIFTとSURF}
SIFT特徴量を算出するアルゴリズムは大きく分けると「キーポイント検出」と「特徴記述」の2つに分けられる。\\
「キーポイント検出」では、Laplacian-of-Gaussian(LoG)を近似したDifference-of-Gaussian(DoG)を用いたスケール探索と特徴点の検出を同時に行い、LoGと比べ計算コストを下げられている。また、DoGの問題点となる処理できない端領域の問題は画像のダウンサンプリングや平滑化画像により解決できる。\\
「特徴記述」では、キーポイントのオリエンテーションを算出と特徴量の記述をしており、回転不偏な特徴量を算出している。
つまり、以上のことからSIFTはスケール変化と回転に対して不変性を持っている特徴量であるということがわかる。
SURFはSIFTの高速化アプローチしたものであり、Boxfilterによる近似や積分画像によって近似ヘッセ行列を算出している.そのため、SIFTではフィルタのサイズによってスケールスペースの処理時間が増加するが、SURFは積分画像により算出時間はフィルタに依存しなくなる.	